\documentclass[a4paper,10pt,draft]{article}
\usepackage[margin=1in]{geometry}
\title{
	{\small
	The University of Melbourne\\
	COMP300024 Artifical Intelligence\\[0.2em]
	}
    Project Part A\\[0.2em]
    \small Searching\\
}
\author{
    James Taranto: 640092
}

\begin{document}
\maketitle
\section{The Search Problem}
%How have you formulated the game as a search problem?
%(You could discuss how you view the problem interms of states, actions, goal tests, and path costs, for example.)
The game has been interpreted as follows:
\begin{description}
	\item[States] location of all pieces in play.
	\item[Actions] legal moves that exist for a state.
	\item[Goal tests] the empty set state, all pieces have left the board.
	\item[Path costs] one for each turn taken (a move, jump or exit action).
\end{description}
\section{The Algorithm}
%What search algorithm does your program use to solve this problem, and why did you choose this algorithm?
%(You  could  comment  on  the  algorithm’s  efficiency,  completeness,  and  optimality.
%You  could  explain  any heuristics you may have developed to inform your search, including commenting on their admissibility.)
A* was decided to be used for this problem for the following reasons:
\begin{description}
	\item[Efficiency] not a strong point for A*, storing every node in memory until a solution is found.
		However, due to the relatively low complexity of the problem, it was deemed suitable for the task.
		By generating a dictionary of heuristic values and available actions for all tiles on the board,
		considerable time has been saved by reusing stored data rather than calculating at each expansion,
		reducing my mean time for solving most test cases to less than a second.
	\item[Completeness] it is complete and will always find a solution if one exists.
	\item[Optimality] given the heuristic is admissible, it will always find the optimal solution.
\end{description}
\subsection{Heuristic}
The heuristic used to inform A* is an adaptation of the manhattan distance to suit a hexagonal grid.
It is then manipulated to return the shortest number of moves required to get a piece to the nearest exit and exiting.
This is calculated with $\lceil\frac{1}{2}d(x)\rceil + 1$, where $d(x)$ is the adapted manhattan distance.
The factor of $\frac{1}{2}$ represents the ability for pieces to jump, effectively halving the distance, except for when the distance required to travel is odd, so we take the ceiling of this.
The constant, 1, is the cost of a piece taking a turn to exit from a goal.
Finally, the heuristic for a given state is given as a sum of the heuristic for each piece to it's nearest goal.\\
This means that the heuristic is admissible, since it never overestimates the cost of reaching the goal state.
\\
\section{Time and Space Requirements}
%What features of the problem and your program’s input impact your program’s time and space requirements?
%(You might discuss the branching factor and depth of your search tree, and explain any other features of theinput which affect the time and space complexity of your algorithm.)
%what is the branching factor here? what is the relative error in h? how are they affected by the location of pieces and blocks?)
Due to the nature of A* (all nodes kept in memory), an apt heuristic is fundamental in ensuring the space complexity is not excessive. 
\begin{description}
	\item[Branching factor] The branching factor is estimated to be $n\cdot \frac{184}{37} \approx n \cdot 4.97$, where $n$ is the number of pieces in play.
		With a board of 37 tiles, there are 184 total moves for a piece.
		This is of course changed by the input, increasing with $n$ and decreasing with the number of blocks.
	\item[Depth] as the complexity of the input increases, the depth of the search tree increases exponentially
	\item[Heuristic] since the heuristic calculates the lowest potential cost and, the relative error can be quite high, since it relies on a piece jumping almost all the way to the exit. 
\end{description}

\end{document}
